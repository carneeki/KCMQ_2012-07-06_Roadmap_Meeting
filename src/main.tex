%-----------------------------------------------------------------------------%
% Format and styling in this file originally created by 
% Carl E. Svensson 2010, updated by Adam J. Carmichael 2011,2012
%-----------------------------------------------------------------------------%
%
%- Document Makeup -----------------------------------------------------------%
%- (01) Notes from template author
%- (02) Document Class and Options
%- (03) Standard package includes and options
%- (04) Custom Definitions and Alterations
%- (05) Custom Commands
%- (06) Document title and other metadata
%- (07) Start Document Content
%- (07a) Misc Config
%
%
%- (01) Notes from carneeki@ -------------------------------------------------%
% NEATNESS:
%  Please keep the TeX neat. Best ways to do this:
%  (01) Don't indent
%  (02) Keep inside of 80 characters (it makes for nicer editing on small
%       laptops).
%  (03) Avoid whitespace between \section{} and other document elements. We
%       have %%%% comments for a reason!
%  (04) Use 2 (that's TWO) space characters to indent. NEVER use tab unless
%       your editor converts to to space chars.
%  (05) Maintain customisations in their respective sections.
%  (06) Comment everything. Bandwidth and diskspace are cheap these days, and
%       TeX compresses pretty nice. Anything else is the BAD kind of laziness
%       on your part.
%
% MULTILINE EQUATIONS:
%  Use \begin{align} instead of \begin{eqnarray}...
%  Details as to why are found at (tl;dr : it's just better...):
%  http://texblog.net/latex-archive/maths/eqnarray-align-environment/
% 
% BIBLIOGRAPHY: 
%  -> URLS: to generate the GUID for a reference that is for a URL, paste
%     the URL into goo.gl and then take only the suffix portion.
%
%  -> Wikipedia citations, simply copy + paste the citation from the
%     menu on the LHS.
%
%
\input{preamble}
%
%- (06) Document Title and other metadata ------------------------------------%
\thiswatermark{
% challenge accepted guy
% bottom right: (220,-760)
% bottom left: (-60,-760)
\put(-60,-760){
\begin{tikzpicture}
  \node[opacity=0.1]{\includegraphics[scale=1.8]{img/challenge.jpg}};
\end{tikzpicture}
}
}
\title{
  KCMQ: Knack Central at Macquarie University \\
  Society Roadmap Meeting: 2012-07-06
}
%
\author{
  Adam J. Carmichael \\
  Department of Engineering\\
  Macquarie University\\
  Sydney, Australia 2109\\
  Email: \url{carl.svensson@ieee.org} \\
  Email: \url{adam.carmichael@ieee.org}
}% author END Brace
%
%- (07) Start Document Content -----------------------------------------------%
\begin{document}
%- (07a) Misc Config ---------------------------------------------------------%
%
\maketitle
%
%\begin{abstract}
%This document is a guide written primarily by a MATH130 student and his friend
%in the 2 week study period between end of class and the examination of
%semester 1, 2011. It is an interpretation that aims to make the very thorough
%notes of Chen and Duong easier and more accessible to the rest of us.
%\end{abstract}
\tableofcontents
\chapter{}
\vspace*{\fill}
\begin{center}
This page is intentionally left blank.
\end{center}
\vspace*{\fill}
\section{Introduction}
\label{sec:Introduction}
\subsection{Attendees}
(in order of surname)
\begin{itemize}
  \item Tristan Allanson (TA)
  \item Tim Boye (TB)
  \item Joe Campbell (JC)
  \item Adam Carmichael (AC), (minutes)
  \item Adrian Kane (AK)
\end{itemize}

\subsection{Apologies}
\begin{itemize}
  \item Pierce Rixon
  \item Carl Svensson
\end{itemize}

\subsection{Agenda}
The following agenda items need to be discussed with the following priority: 
\begin{enumerate}
  \item Whats - Determine what we intend on doing
  \item Whens - When we intend on doing them. Would like to determine (to the
  week) when activities and events will take place
  \item Whos - Who will take point on activity / event / item 
  \item Hows - How the activity / event / item will be implemented
\end{enumerate}
\include{chaps/01_name}

\end{document}